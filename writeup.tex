\documentclass[titlepage]{scrartcl}
\usepackage{graphicx}
\addtokomafont{disposition}{\rmfamily}
%opening
\titlehead{BIOL 224L}
\title{Assignment 4}
\date{\today}
\author{Dean Pearce}

\begin{document}
\maketitle
\tableofcontents
\clearpage
\section{Explanation of mutation code}
\subsection{Code}
\begin{verbatim}
    % set variables
    prob_mutation = 1e-8;
    cultures = 30;
    generations = 20;
    initial_inoculum = 100;
    
    mut=zeros(cultures,1);
    for i=1:cultures
        n=initial_inoculum;
        for t=1:generations
            n=2*n;
            new_mutations = poissrnd(n*prob_mutation);
            mut(i) = 2*mut(i)+new_mutations;
            n = n-new_mutations;
        end
    end
\end{verbatim}
\subsection{Explanation}
\subsubsection{Initial Parameters}
\begin{description}
    \item[prob\_mutation] The probability of a sensitive bacteria to mutate to resistant
    \item[cultures] The number of separate, discrete pools of bacteria
    \item[generations] The number of generations the simulation will run
    \item[initial\_inoculum] The initial number of bacteria in each culture (assumed to all be sensitive)
\end{description}
\subsubsection{Further Variables}
\begin{description}
    \item[i] Index of cultures
    \item[t] Index of generations
    \item[n] Number of sensitive bacteria in each culture
    \item[mut] 1d array of number of mutants in each culture
    \item[new\_mutations] Number of new mutants created in a generation
\end{description}
\subsection{Code Analysis}
\begin{description}
    \item[Lines 2-5] Instantiate initial variables
    \item[Line 7] Initialize \textbf{mut} with all 0s
    \item[Lines 8-16] For each culture:
    \begin{description}
        \item[Line 8] Set \textbf{n} equal to initial inoculum
        \item[Lines 9-15] For each generation:
        \begin{description}
            \item[Line 11] Double \textbf{n} (mitosis)
            \item[Line 12] Calculate number of mutants (via $\textbf{prob\_mutation}*\textbf{n}$)
            \item[Line 13] Double number of mutants (\textbf{mut}) and add new mutants to mutant population (\textbf{mut})
            \item[Line 14] Remove new mutants from sensitive population (\textbf{n})
        \end{description}
    \end{description}
\end{description}
\section{Histogram and Fano Factor}
\subsection{Histogram}
\begin{figure}[h]
    \centering
    \includegraphics[scale=0.75]{Problem2.png}
    \caption{Simulated distribution of resistant bacteria in each culture for mutation hypothesis}
    \label{fig:problem2_histogram}
\end{figure}
\subsection{Fano Factor Calculation}
\subsubsection{Code}
\begin{verbatim}
    ...
    fano_factor_mut = var(mut)/mean(mut);    
\end{verbatim}
\subsubsection{Results}
The Fano Factor for the same population as Figure \ref{fig:problem2_histogram} is $58.8367589604213$
\end{document}