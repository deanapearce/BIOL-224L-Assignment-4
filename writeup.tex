\documentclass[titlepage]{scrreprt}
\usepackage{graphicx}
\usepackage{float}
\usepackage{amsmath}
\addtokomafont{disposition}{\rmfamily}
%opening
\titlehead{BIOL 224L}
\title{Assignment 4}
\date{\today}
\author{Dean Pearce}

\begin{document}
\maketitle
\tableofcontents
\clearpage
\chapter{Mutation Hypothesis}
\section{Explanation of mutation code}
\subsection{Code}
\begin{verbatim}
    % set variables
    prob_mutation = 1e-8;
    cultures = 30;
    generations = 20;
    initial_inoculum = 100;
    
    mut=zeros(cultures,1);
    for i=1:cultures
        n=initial_inoculum;
        for t=1:generations
            n=2*n;
            new_mutations = poissrnd(n*prob_mutation);
            mut(i) = 2*mut(i)+new_mutations;
            n = n-new_mutations;
        end
    end
\end{verbatim}

\subsection{Initial Parameters}
\begin{description}
    \item[prob\_mutation] The probability of a sensitive bacteria to mutate to resistant
    \item[cultures] The number of separate, discrete pools of bacteria
    \item[generations] The number of generations the simulation will run
    \item[initial\_inoculum] The initial number of bacteria in each culture (assumed to all be sensitive)
\end{description}
\subsection{Additional Variables}
\begin{description}
    \item[i] Index of cultures
    \item[t] Index of generations
    \item[n] Number of sensitive bacteria in each culture
    \item[mut] 1d array of number of mutants in each culture
    \item[new\_mutations] Number of new mutants created in a generation
\end{description}
\subsection{Code Analysis}
\begin{description}
    \item[Lines 2-5] Instantiate initial variables
    \item[Line 7] Initialize \textbf{mut} with all 0s
    \item[Lines 8-16] For each culture:
    \begin{description}
        \item[Line 8] Set \textbf{n} equal to initial inoculum
        \item[Lines 9-15] For each generation:
        \begin{description}
            \item[Line 11] Double \textbf{n} (mitosis)
            \item[Line 12] Calculate number of mutants (via $\textbf{prob\_mutation}*\textbf{n}$)
            \item[Line 13] Double number of mutants (\textbf{mut}) and add new mutants to mutant population (\textbf{mut})
            \item[Line 14] Remove new mutants from sensitive population (\textbf{n})
        \end{description}
    \end{description}
\end{description}
\section{Histogram and Fano Factor}
\subsection{Histogram}
\begin{figure}[H]
    \centering
    \includegraphics[scale=0.75]{Problem2.png}
    \caption{Simulated distribution of resistant bacteria in each culture for mutation hypothesis}
    \label{fig:problem2_histogram}
\end{figure}
\subsection{Fano Factor Calculation}
\subsubsection{Code}
\begin{verbatim}
    ...
    fano_factor_mut = var(mut)/mean(mut);    
\end{verbatim}
\subsubsection{Results}
The Fano Factor for the same population as Figure \ref{fig:problem2_histogram} is $58.8367589604213$
\chapter{Acquired Immunity Hypothesis}
\section{Code Explanation}
\subsection{Code}
\begin{verbatim}
%set variables
params;

acq = zeros(cultures,1);
n = initial_inoculum*2^generations;
for i = 1:cultures
    acq(i) = binornd(n,prob_mutation);
end
\end{verbatim}
\subsection{Initial Parameters}
\begin{description}
    \item[prob\_mutation] The probability of a sensitive bacteria to mutate to resistant
    \item[cultures] The number of separate, discrete pools of bacteria
    \item[generations] The number of generations the simulation will run
    \item[initial\_inoculum] The initial number of bacteria in each culture (assumed to all be sensitive)
\end{description}
\subsection{Additional Variables}
\begin{description}
    \item[acq] Number of mutant bacteria in each culture
    \item[n] Number of sensitive bacteria in each culture
    \item[i] Index of cultures 
\end{description}
\subsection{Code Analysis}
\begin{description}
    \item[Line 1] Initialize parameters
    \item[Line 3] Initialize number of mutant bacteria (\textbf{acq}) to zero
    \item[Line 4] Calculate total number of bacteria (all sensitive) in final generation prior to exposure to challenge
    \item[Line 5-7] For each culture:
    \begin{description}
        \item[Line 6] Calculate number of mutants via binomial distribution from number of bacteria prior to challenge (\textbf{n}) and mutation probability (\textbf{prob\_mutation})
        \begin{quote}
            Matlab Reference: r = binornd(n,p) generates random numbers from the binomial distribution specified by the number of trials n and the probability of success for each trial p
        \end{quote}
    \end{description}
\end{description}
\section{Histogram and Fano Factors}
\subsection{Histogram}
\begin{figure}[H]
    \centering
    \includegraphics[scale=0.75]{Problem3.png}
    \caption{Simulated distribution of resistant bacteria in each culture for acquired immunity hypothesis}
    \label{fig:problem3_histogram}
\end{figure}
\subsection{Fano Factors}
\subsubsection{Results}
The Fano factor for the same population as Figure \ref{fig:problem3_histogram} is $1.26840633737185$
\subsubsection{Interpretation}
The Fano factor for the acquired immunity hypothesis is substantially lower than that of the mutation hypothesis.  This was to be expected, as the expected mean for the mutation hypothesis is much higher than that for acquired immunity as a result of the possibility of "jackpots", where an early mutation causes a large number of resistant bacteria.  This would, of course, increase the variance substantially.  However, the acquired immunity hypothesis has a far lower mean, and a variance which is not low enough to increase the Fano factor to a great degree.
\section{Binomial v. Possion}
As the number of mutants is a result of a number of independent trials with one single probability, it is possible to sample from a binomial distribution a great number of times.  This is exactly what the Poisson distribution represents in fact.
\subsection{Code Rewrite}
\subsubsection{Rewrite}
\begin{verbatim}
    acq = zeros(cultures,1);
    n = initial_inoculum*2^generations;
    for i = 1:cultures
        acq(i) = poissrnd(n*prob_mutation);
    end
\end{verbatim}
\subsubsection{Results}
\begin{figure}[H]
    \centering
    \includegraphics[scale=0.75]{Problem4.png}
    \caption{Simulated distribution of resistant bacteria in each culture for acquired immunity hypothesis, calculated via Poisson distribution}
    \label{fig:problem4_histogram}
\end{figure}
The Fano factor for the same population as Figure \ref{fig:problem4_histogram} is $1.38793103448276$.
\chapter{Comparisons}
\section{Mutation Rate Estimations}
\subsection{Acquired Resistance}
\subsubsection{Calculation}
\begin{equation*}
\begin{split}
    \frac{11}{30} &= e^{-m} \\
    m &\approx 1.0033 \\
    m &= a(N_t-N_0) \\
    1.0033 &= a*(100*2^{20}-100) \\
    a &= 9.56822*10^-9
\end{split}
\end{equation*}
\subsubsection{Analysis}
This number is, of course, fairly similar to the actual mutation rate $1*10^{-8}$.  Since, effectively, only a single generation of mutation was simulated, and with a very high sample size, this is a logical outcome.
\subsection{Mutation}
\subsubsection{Calculations}
\begin{equation*}
    \begin{split}
        \frac{24}{30} &= e^{-m} \\
        m &\approx 0.223144 \\
        m &= a(N_t-N_0) \\
        0.223144 &= a*(100*2^{20}-100) \\
        a &= 2.12807*10^-9 %\text{\,per time cycle (20 generations)}
    \end{split}
\end{equation*}
\subsubsection{Analysis}
This estimated mutation rate is much lower than the actual mutation rate $1*10^{-8}$.  This is for a number of reasons, the primary reason being that this method only takes into account a single time unit and utilizes only the proportion of sensitive bacteria to resistant.
\section{Experimental Design}
\subsection{Parameter space for indistinguishable models}
\subsubsection{Hypothesis}
I would hypothesize that a region of parameter space which would be difficult to distinguish would be very low number of generations, and fairly low mutation rate, though this is not as critical.
\subsubsection{Testing}
The following parameters were used:
\begin{verbatim}
    prob_mutation = 1e-8;
    cultures = 30;
    generations = 2;
    initial_inoculum = 100;
\end{verbatim}
\subsubsection{Results}
\begin{figure}[H]
    \centering
    \includegraphics[scale=0.75]{Problem6_mut.png}
    \caption{Simulated distribution of resistant bacteria in each culture for mutation hypothesis, using modified parameters}
    \label{fig:problem6_mut_histogram}
\end{figure}
\begin{figure}[H]
    \centering
    \includegraphics[scale=0.75]{Problem6_acq.png}
    \caption{Simulated distribution of resistant bacteria in each culture for acquired immunity hypothesis, using modified parameters}
    \label{fig:problem6_acq_histogram}
\end{figure}
The Fano factors for both populations were incomputable.
\subsubsection{Analysis}
The results did indeed support my hypothesis, as with such low mutation rates, no mutations arose in either model.  In retrospect, this should have been trivial.
\end{document}